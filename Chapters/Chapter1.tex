% Chapter 1

\chapter{Introduction} % Write in your own chapter title
\label{Chapter1}
\lhead{Chapter 1. \emph{Introduction}} % Write in your own chapter title to set the page header


\section{Neutron Interferometry}
\subsection{History}
Interferometry has long been a powerful tool for experimental physics. Its various forms have been used in the discovery of many historically significant results such as the Michelson-Morley experiment which showed that the speed of light was independent of inertial reference frames and experimental data in support of Bell's Inequality. \cite{michelson_morley}\cite{bells_inequality}

The key concept of interferometry is the superposition of waveforms upon each other in order to deduce meaningful physical properties from the resultant combination. If one considers two waves of identical frequency than the waves when superimposed will combine constructively when in phase and de-constructively when out of phase. The technique of interferometry can be applied to many different experimental systems, the requirement being that the interferometry medium be described as a wave mathematically. Such systems that have been used in the past include electromagnetic waves, water waves, electrons and neutrons. Although electrons and neutrons classically are described as point particles the development of quantum mechanics allows that all matter is actually described by a waveform and therefore interferometry techniques may be applied to the electron and neutron waveforms. This paper focuses primarily on neutron interferometry. 

The first Neutron Interferometer with slow neutrons was constructed by Maier-Leibnitz and Springer in 1962 and was effectively equivalent to a double slit experiment. However, their interferometer was not effective for measuring physical properties of materials. In 1965 the perfect single-crystal interferometer was theorized by Ulrich Bonse and Michael Hart, however it was not until 1974 that their interferometer was made functional by Helmut Raunch and his student Wolfgang Treimer. Their interferometer used a single perfect crystal in which two horizontal slices were removed from the interior to form a three-blade interferometer.\cite{neutron_history} \textbf{INSERT FIGURE}. Using the single-crystal design researchers Colella,Overhauser and Werner to perform the famous COW experiment which measured the phase shift due to the gravitational potential difference between two neutron beams separated by a small displacement in height.\cite{cow} Further experiments made such contributions to experimental physics such as the measurement of the Aharonov-Bohm effect and the the effect of the Earth's rotation on a quantum system.\cite{neutron_history} It was quickly realized that neutron interferometry measurements provide an incredible level of accuracy and isolation in experimental measurements. This is due to the fact that the neutron has essentially zero electric charge and therefore does not feel the Coulomb force. Therefore for the case of slow neutrons there is no need to isolate for stray electric fields. 
\subsection{Application to Quantum Information}
As the neutron interferometry provides a low-noise experimental system it provides an ideal test-bench for testing certain aspects of quantum information theory. Such an example was the use of a five-blade interferometer which allowed the quantum information encoded in the neutron waveform by using additional blades to exploit the symmetry of mechanical vibrations in the interferometer and decouple these modes.\cite{five_blade}. This is an example of encoding the information into a decoherence-free subspace and is a technique that may be applicable in future scalable quantum computation systems. Additionally it has been shown that neutron interferometers can be used for the generation of single neutron entangled states. \cite{neutron_entanglement}   Additionally there is interest in the quantum discord of neutron interferometry systems and there application towards non-classical discord algorithms.\cite{noise_neutron}. It is unlikely that a scalable quantum computer will be realizable with neutrons due to their low interaction with other quantum systems.  
\subsection{Application to Quantum Fundamentals}
Neutron interferometry has played a large role in experimentally gathering information on the fundamental behaviour of quantum systems. Such as the Aharonov-Bohm effect, the effect of gravity,quantum discord and verifying Bell's Inequality. \cite{neutron_history}\cite{cow}\cite{noise_neutron}\cite{bells_inequality}. More recently researchers at the Institute for Quantum Computing are designing an experimental neutron interferometer that is equivalent to a triple-slit experiment in the search for third order interference effects that are theoretically non-existent but if found may be evidence of new quantum theories.\cite{three_slit} 
\subsection{National Institute of Standards and Technology}
\section{Bayesian Markov Chain Monte Carlo Methods}

