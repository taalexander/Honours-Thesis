% Chapter 4

\chapter{Experimental Setup} % Write in your own chapter title
\label{Chapter4}
\lhead{Chapter 4. \emph{Experimental Setup}} % Write in your own chapter title to set the page header


\section{The Neutron Interferometer}
The neutron interferometer that this thesis refers to is located at the Neutron Interferometry and Optics facility (NIOF) at the National Institute of Standards and Technology (NIST) in Gaithersburg, MD. 
\subsection{NIST}
\subsection{Reactor}
NIST operates a 20MW split-core research reactor. Neutrons of approximate energy $1 MeV$ are emitted during $^{235}U$ fission and then thermalized using heavy water ($D_2O$) as a moderator. This brings the neutrons to room temperature as discussed in (\ref{sec:thermal}). At the reactor core the peak thermal neutron flux is $4\times 10^{14} neutrons/cm^2$. The reactor is operated on a seven week cycle during which it is operated at full power for 38 days and then followed by 11 days of refuelling and maintenance operations. 

As the longer wavelength of cold neutrons ($\lambda>1.8\AA$ and $E<25meV$) is often desired for condensed matter study there is a cold moderator installed next to the core. The thermal neutrons scatter with liquid hydrogen at $20K$ and exit with a Maxwellian distribution of characteristic temperature of $34K$. 

There are eight thermal neutron ports available for lab use. The neutrons are transported to the instruments in the NCNR hall using neutron guides. The neutron interferometer facility is located on the NG7 guide shown in figure (\ref{fig:guides}). The guides are of a rectangular cross-section and are produced by gluing together meter long sections of $100nm$ thick $^{58}Ni$ optically-flat borated glass plates. $^{58}Ni$ is used due to its large neutron reflective potential. 
$$V = \frac{2\pi*\hbar^2}{m}\rho=\frac{2\pi\hbar^2}{m}\frac{1}{V}\sum\limits_{i} b = 335neV$$
If the perpendicular component of the neutron energy incident on the guide is less than the potential of the guide it will be reflected. 


\subsection{Motors and Actuators}
\subsection{Sensors}
\section{NI-Engine}
\subsection{Design Requirements}
\subsection{Language and Library Choices}
\subsection{System Architecture}
\subsection{Documentation}
\section{Q-Infer}
\subsection{Interaction with NI-Engine}
\subsection{GPU Implementations of Likelihood functions}

